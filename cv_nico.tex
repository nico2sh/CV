%%%%%%%%%%%%%%%%%
% This is an example CV created using altacv.cls (v1.1.3, 30 April 2017) written by
% LianTze Lim (liantze@gmail.com), based on the
% Cv created by BusinessInsider at http://www.businessinsider.my/a-sample-resume-for-marissa-mayer-2016-7/?r=US&IR=T
%
%% It may be distributed and/or modified under the
%% conditions of the LaTeX Project Public License, either version 1.3
%% of this license or (at your option) any later version.
%% The latest version of this license is in
%%    http://www.latex-project.org/lppl.txt
%% and version 1.3 or later is part of all distributions of LaTeX
%% version 2003/12/01 or later.
%%%%%%%%%%%%%%%%

%% If you want to use \orcid or the
%% academicons icons, add "academicons"
%% to the \documentclass options.
%% Then compile with XeLaTeX or LuaLaTeX.
% \documentclass[10pt,a4paper,academicons]{altacv}

%% Use the "normalphoto" option if you want a normal photo instead of cropped to a circle
% \documentclass[10pt,a4paper,normalphoto]{altacv}

\documentclass[10pt,a4paper]{altacv}

%% AltaCV uses the fontawesome and academicon fonts
%% and packages.
%% See texdoc.net/pkg/fontawecome and http://texdoc.net/pkg/academicons for full list of symbols.
%% When using the "academicons" option,
%% Compile with LuaLaTeX for best results. If you
%% want to use XeLaTeX, you may need to install
%% Academicons.ttf in your operating system's font %% folder.


% Change the page layout if you need to
\geometry{left=1cm,right=9cm,marginparwidth=6.8cm,marginparsep=1.2cm,top=0.4cm,bottom=0.4cm}

% Change the font if you want to.

% If using pdflatex:
\usepackage[utf8]{inputenc}
\usepackage[T1]{fontenc}
\usepackage[default]{lato}

% If using xelatex or lualatex:
% \setmainfont{Lato}

% Change the colours if you want to
\definecolor{VividPurple}{HTML}{3E0097}
\definecolor{SlateGrey}{HTML}{2E2E2E}
\definecolor{LightGrey}{HTML}{666666}
\colorlet{heading}{VividPurple}
\colorlet{accent}{VividPurple}
\colorlet{emphasis}{SlateGrey}
\colorlet{body}{LightGrey}

% Change the bullets for itemize and rating marker
% for \cvskill if you want to
\renewcommand{\itemmarker}{{\small\textbullet}}
\renewcommand{\ratingmarker}{\faCircle}

%% sample.bib contains your publications
\addbibresource{sample.bib}

\begin{document}
\name{Nicolás Hormazábal}
\tagline{Software Engineer \& Geek}

\photo{2.5cm}{nico}
\personalinfo{%
  % Not all of these are required!
  % You can add your own with \printinfo{symbol}{detail}
  \email{mail@nico.red}
  \phone{+34 666134936}
  \mailaddress{C/ Marqués Caldes de Montbui 27 3-2, 17003}
  \location{Girona, Spain}
  \homepage{nico.red}
  \twitter{@pnikosis}
  \linkedin{linkedin.com/in/nhormazabal}
  \github{github.com/pnikosis}
%   \orcid{orcid.org/0000-0000-0000-0000} % Obviously making this up too. If you want to use this field (and also other academicons symbols), add "academicons" option to \documentclass{altacv}
}

%% Make the header extend all the way to the right, if you want.
\begin{fullwidth}
\makecvheader
\end{fullwidth}

%% Provide the file name containing the sidebar contents as an optional parameter to \cvsection.
%% You can always just use \marginpar{...} if you do
%% not need to align the top of the contents to any
%% \cvsection title in the "main" bar.
\cvsection[nico-p1sidebar]{Experience}

\cvevent{Engineering Manager}{Schibsted Product \& Technology}{Nov 2016 -- Ongoing}{Barcelona, Spain}
\begin{itemize}
	\item Led the Notifications team to develop a mature product, used in several sites across the company for delivering notifications with high engagement impact.
	\item Contributed and lead experimentation to develop new features in collaboration with other teams.
\end{itemize}

\divider

\cvevent{Tech Lead}{Schibsted Product \& Technology}{Nov 2015 -- Nov 2016}{Barcelona, Spain}
\begin{itemize}
	\item Design, develop and coordinate the Notifications team to create a scalable, performing service.
	\item Defined and prioritized the product requirements for the notifications service defining its requirements and applying an integration roadmap.
\end{itemize}

\divider

\cvevent{Android Tech Lead}{Schibsted Product \& Tech}{March 2014 -- Nov 2015}{Barcelona, Spain}

\begin{itemize}
	\item Lead the Android Mobile team to develop the NextGen app.
	\item Coordinated the different local development teams for the integration and branding of the NextGen app for the local markets.
\end{itemize}

\divider	

\cvevent{Software Engineer}{Freelance}{June 2013 -- Feb 2014}{Barcelona/Girona, Spain}

\begin{itemize}
	\item Software development as a freelance. Mostly native mobile apps on Android and interactive software.
	\item Developed a custom personalization layer for a family friendly tablet (for FACTIS/Milan: https://www.milan.es/).
\end{itemize}

\divider	

\cvevent{Software Engineer/Researcher}{Centre Easy Innova/Universitat de Girona/Fatronik-Tecnalia}{Apr 2008 -- Jun 2013}{Girona/Gipúzkoa, Spain}

\begin{itemize}
	\item Several tasks as a researcher and software engineer during a PhD process.
	\item Main research topic in Artificial Intelligence/Virtual Organizations and Trust \& Reputation systems.
\end{itemize}

\divider

\cvevent{Software Developer}{Freelance/Inoxpa}{2003 -- 2008}{Girona, Spain}

\begin{itemize}
	\item Several small software (web and desktop) development projects.
\end{itemize}

\cvsection{A Regular Week}

% Adapted from @Jake's answer from http://tex.stackexchange.com/a/82729/226
% \wheelchart{outer radius}{inner radius}{
% comma-separated list of value/text width/color/detail}
\wheelchart{1.3cm}{0.3cm}{%
  5/12em/accent!10/Cooking for the family,
  20/14em/accent!50/Sync with the team and make sure they feel motivated,
  10/14em/accent!20/Comics videogames and other non-grown up stuff,
  20/12em/accent!40/My own kids!,
  15/12em/accent!30/Define roadmap sync with product and plan,
  10/10em/accent!60/Catching up with tech on news and learning new stuff,
  10/12em/accent/Review the team's work,
  10/14em/accent!30/Actual software development
}



\end{document}
